%--------This section sets up the document class and packages.
\usepackage[includeheadfoot, margin=20mm, headheight=20mm]{geometry} 
\usepackage{fancyhdr}
\usepackage{mwe}
\usepackage{amsmath}
\usepackage{amsthm}
\usepackage[utf8]{inputenc}
\usepackage{amssymb}
%\usepackage{xcolor,graphicx}
\usepackage{hyperref}
\usepackage{centernot}
\usepackage{tikz}  % Uncomment this line iff you are using the tikz package to add drawings
\usepackage{tkz-euclide}
\usepackage{pgf}
\usepackage{pgfplots}
\pgfplotsset{compat=newest}
\pgfplotsset{plot coordinates/math parser=false}
\usepackage[toc,page]{appendix}
\usepackage{hyperref}
\usepackage{pdfpages}
\usepackage{xargs}                      % Use more than one optional parameter in a new commands
%\usepackage[pdftex,dvipsnames]{xcolor}  % Coloured text etc.
% 
\usepackage[colorinlistoftodos,prependcaption]{todonotes}
\newcommandx{\unsure}[2][1=]{\todo[linecolor=red,backgroundcolor=red!25, size=\small,bordercolor=red,#1]{#2}}


\allowdisplaybreaks
\pgfplotsset{soldot/.style={color=blue,only marks,mark=*}} \pgfplotsset{holdot/.style={color=blue,fill=white,only marks,mark=*}}

\usepackage{tocloft}
\renewcommand{\cftsecleader}{\cftdotfill{\cftdotsep}}
\renewcommand\labelenumi{(\roman{enumi})}

%--- This section makes possible customized theorem numbering.
\newtheorem{innercustomgeneric}{\customgenericname}
\providecommand{\customgenericname}{}
\newcommand{\newcustomtheorem}[2]{%
  \newenvironment{#1}[1]
  {%
   \renewcommand\customgenericname{#2}%
   \renewcommand\theinnercustomgeneric{##1}%
   \innercustomgeneric
  }
  {\endinnercustomgeneric}
}
\newcustomtheorem{cthm}{Theorem}
\newcustomtheorem{caxm}{Axiom}
\newcustomtheorem{clem}{Lemma}
\newcustomtheorem{ccor}{Corollary}
\newcustomtheorem{cprop}{Proposition}
\newcustomtheorem{cdefn}{Definition}
\newcustomtheorem{ceg}{Example}
\newcustomtheorem{crmk}{Remark}
\newcustomtheorem{ccus}{}
\newcustomtheorem{cprob}{Problem}

%---The following code sets up the way theorems are typeset and labeled.
\newtheorem{thm}{Theorem}
\newtheorem{lem}[thm]{Lemma}
\newtheorem{cor}[thm]{Corollary}
\newtheorem{prop}[thm]{Proposition}
\theoremstyle{definition}
\newtheorem{defn}[thm]{Definition}
\newtheorem{axm}[thm]{Axiom}
\newtheorem{eg}[thm]{Example}
\theoremstyle{remark}
\newtheorem{rmk}[thm]{Remark}
\newtheorem{sol}{Solution}

\numberwithin{equation}{subsection}
\numberwithin{thm}{subsection}

% \newtheorem*{thm}{Theorem}
% \newtheorem*{lem}{Lemma}
% \newtheorem*{cor}{Corollary}
% \newtheorem*{prop}{Proposition}
% \theoremstyle{definition}
% \newtheorem*{defn}{Definition}
% \newtheorem*{axm}{Axiom}
% \newtheorem*{eg}{Example}
% \theoremstyle{remark}
% \newtheorem*{rmk}{Remark}
% \newtheorem*{sol}{Solution}


%---The following code defines a few extra commands that will be useful in some exercises.
\newcommand{\abs}[1]{\lvert#1\rvert}     % Absolute value symbol
\newcommand{\Abs}[1]{\Bigg\lvert#1\Bigg\rvert}     % Absolute value symbol (big)
\newcommand{\Z}{\mathbb Z}              % The set of integers
\newcommand{\Q}{\mathbb Q}              % The set of rationals
\newcommand{\R}{\mathbb R}              % The set of reals
\newcommand{\N}{\mathbb N}              % The set of natural numbers
\newcommand{\C}{\mathbb C}              % The set of complex numbers
\newcommand{\F}{\mathbb F}  

\newcommand{\ZZ}{\mathcal{Z}}
\newcommand{\OO}{\mathcal{O}}
\newcommand{\CC}{\mathcal{C}}
\newcommand{\UU}{\mathcal{U}}
\newcommand{\power}{\mathcal{P}}         % The power set of a set
\newcommand{\bfun}{\mathcal{F}}          % The finite subsets
\newcommand{\Id}{\mathrm{Id}}            % The identity function
\newcommand{\nil}{\emptyset}             % Empty set
\newcommand{\inflim}[1]{\lim_{#1\to\infty}} % Limit to infinity
\newcommand{\ninflim}[1]{\lim_{#1\to\infty}}% Limit to negative infinity
\providecommand{\BVec}[1]{\mathbf{#1}}   % Bold font for vectors
\DeclareMathOperator{\gon}{gon}          % A polygon
\DeclareMathOperator{\Fun}{Fun}          % The set of all functions from one set to another
\DeclareMathOperator{\Perm}{Perm}        % The set of all permutations on a set
\DeclareMathOperator{\Int}{int} %interior of a set
\DeclareMathOperator{\cl}{cl} %closure of a set
\DeclareMathOperator{\diam}{diam}
\DeclareMathOperator{\sinc}{sinc}
\DeclareMathOperator{\rank}{rank}
%\newcommand{\Re}{\mathrm{Re}} %real part