\documentclass[10.5pt, oneside, a4paper]{article}
%--------This section sets up the document class and packages.
\usepackage[includeheadfoot, margin=20mm, headheight=20mm]{geometry} 
\usepackage{fancyhdr}
\usepackage{mwe}
\usepackage{amsmath}
\usepackage{amsthm}
\usepackage[utf8]{inputenc}
\usepackage{amssymb}
%\usepackage{xcolor,graphicx}
\usepackage{hyperref}
\usepackage{centernot}
\usepackage{tikz}  % Uncomment this line iff you are using the tikz package to add drawings
\usepackage{tkz-euclide}
\usepackage{pgf}
\usepackage{pgfplots}
\pgfplotsset{compat=newest}
\pgfplotsset{plot coordinates/math parser=false}
\usepackage[toc,page]{appendix}
\usepackage{hyperref}
\usepackage{pdfpages}
\usepackage{xargs}                      % Use more than one optional parameter in a new commands
%\usepackage[pdftex,dvipsnames]{xcolor}  % Coloured text etc.
% 
\usepackage[colorinlistoftodos,prependcaption]{todonotes}
\newcommandx{\unsure}[2][1=]{\todo[linecolor=red,backgroundcolor=red!25, size=\small,bordercolor=red,#1]{#2}}
\usepackage{url}
\usepackage[ruled,vlined]{algorithm2e}
\SetKwComment{Comment}{$\triangleright$\ }{}
\usepackage{listings}

\allowdisplaybreaks
\pgfplotsset{soldot/.style={color=blue,only marks,mark=*}} \pgfplotsset{holdot/.style={color=blue,fill=white,only marks,mark=*}}

\usepackage{tocloft}
\renewcommand{\cftsecleader}{\cftdotfill{\cftdotsep}}
\renewcommand\labelenumi{(\roman{enumi})}

%--- This section makes possible customized theorem numbering.
\newtheorem{innercustomgeneric}{\customgenericname}
\providecommand{\customgenericname}{}
\newcommand{\newcustomtheorem}[2]{%
  \newenvironment{#1}[1]
  {%
   \renewcommand\customgenericname{#2}%
   \renewcommand\theinnercustomgeneric{##1}%
   \innercustomgeneric
  }
  {\endinnercustomgeneric}
}
\newcustomtheorem{cthm}{Theorem}
\newcustomtheorem{caxm}{Axiom}
\newcustomtheorem{clem}{Lemma}
\newcustomtheorem{ccor}{Corollary}
\newcustomtheorem{cprop}{Proposition}
\newcustomtheorem{cdefn}{Definition}
\newcustomtheorem{ceg}{Example}
\newcustomtheorem{crmk}{Remark}
\newcustomtheorem{ccus}{}
\newcustomtheorem{cprob}{Problem}

%---The following code sets up the way theorems are typeset and labeled.
\newtheorem{thm}{Theorem}
\newtheorem{lem}[thm]{Lemma}
\newtheorem{cor}[thm]{Corollary}
\newtheorem{prop}[thm]{Proposition}
\theoremstyle{definition}
\newtheorem{defn}[thm]{Definition}
\newtheorem{axm}[thm]{Axiom}
\newtheorem{eg}[thm]{Example}
\theoremstyle{remark}
\newtheorem{rmk}[thm]{Remark}
\newtheorem{sol}{Solution}

% \numberwithin{equation}{subsection}
% \numberwithin{thm}{section}

% \newtheorem*{thm}{Theorem}
% \newtheorem*{lem}{Lemma}
% \newtheorem*{cor}{Corollary}
% \newtheorem*{prop}{Proposition}
% \theoremstyle{definition}
% \newtheorem*{defn}{Definition}
% \newtheorem*{axm}{Axiom}
% \newtheorem*{eg}{Example}
% \theoremstyle{remark}
% \newtheorem*{rmk}{Remark}
% \newtheorem*{sol}{Solution}


%---The following code defines a few extra commands that will be useful in some exercises.
\newcommand{\abs}[1]{\lvert#1\rvert}     % Absolute value symbol
\newcommand{\Abs}[1]{\Bigg\lvert#1\Bigg\rvert}     % Absolute value symbol (big)
\newcommand{\Z}{\mathbb Z}              % The set of integers
\newcommand{\Q}{\mathbb Q}              % The set of rationals
\newcommand{\R}{\mathbb R}              % The set of reals
\newcommand{\N}{\mathbb N}              % The set of natural numbers
\newcommand{\C}{\mathbb C}              % The set of complex numbers
\newcommand{\F}{\mathbb F}  

\newcommand{\ZZ}{\mathcal{Z}}
\newcommand{\OO}{\mathcal{O}}
\newcommand{\CC}{\mathcal{C}}
\newcommand{\UU}{\mathcal{U}}
\newcommand{\power}{\mathcal{P}}         % The power set of a set
\newcommand{\bfun}{\mathcal{F}}          % The finite subsets
\newcommand{\Id}{\mathrm{Id}}            % The identity function
\newcommand{\nil}{\emptyset}             % Empty set
\newcommand{\inflim}[1]{\lim_{#1\to\infty}} % Limit to infinity
\newcommand{\ninflim}[1]{\lim_{#1\to\infty}}% Limit to negative infinity
\providecommand{\BVec}[1]{\mathbf{#1}}   % Bold font for vectors
\DeclareMathOperator{\gon}{gon}          % A polygon
\DeclareMathOperator{\Fun}{Fun}          % The set of all functions from one set to another
\DeclareMathOperator{\Perm}{Perm}        % The set of all permutations on a set
\DeclareMathOperator{\Int}{int} %interior of a set
\DeclareMathOperator{\cl}{cl} %closure of a set
\DeclareMathOperator{\diam}{diam}
\DeclareMathOperator{\sinc}{sinc}
\DeclareMathOperator{\rank}{rank}
\DeclareMathOperator{\Span}{span}
\DeclareMathOperator{\len}{length}
%\newcommand{\Re}{\mathrm{Re}} %real part

\providecommand{\LIN}[1]{{\color{blue}#1}} %Linda's notes

% Dave's macros
\definecolor{colorDAS}{RGB}{255,127,0}
\providecommand{\DAS}[1]{{\color{colorDAS}#1}}     % Dave's comments

%---header/style/enumeration-------
\pagestyle{fancy}
\lhead{Capstone}
\chead{MCS}
\rhead{2018-2019 Semester 1}
\author{Linfan}
%----------------------------------

%-----------MY INFORMATION---------
%\title{{\fontfamily{qbk}\selectfont
\title{Reading notes}
\date{\vspace{-5ex}}
%----------------------------------
%unnumbered sections in TOC
%\setcounter{secnumdepth}{0}

%-----This is where the GOOD STUFF begins---
\begin{document}

\maketitle

\thispagestyle{fancy}

This document contains all notes taken while reading materials (e.g., textbooks, literature) in preparation for capstone. Black text are information consolidated from the readings; blue text are notes; red text are questions.

\tableofcontents
\listoftodos

\newpage
\section{\href{./textbooks/[Earl_A_Coddington_Norman_Levinson]_Theory_of_Ordinary_Differential_Equations.pdf}{Theory of Ordinary Differential Equations (Coddington Levinson)}}

\subsection{Chapter 11: Algebraic properties of linear boundary-value problems on a finite interval}

\subsubsection{Introduction}
\begin{defn}
Let $L$ be the \href[page=296]{./textbooks/[Earl_A_Coddington_Norman_Levinson]_Theory_of_Ordinary_Differential_Equations.pdf}{\textbf{linear differential operator of oder $n$ ($n\geq 1$)}} defined by
\[Lx = p_0x^{(n)} + p_1x^{(n-1)} + \cdots + p_{n-1}x' + p_nx\]
where the $p_k$ are complex-valued functions of class $C^{n-k}$ on a closed bounded interval $[a,b]$ {\color{blue}(i.e., derivatives $p_k, p'_k, \ldots, p^{(n-k)}_k$ exist on $[a,b]$ and are continuous)} and $p_0(t)\neq 0$ on $[a,b]$.
\end{defn}

\begin{defn}
    \href[page=296]{./textbooks/[Earl_A_Coddington_Norman_Levinson]_Theory_of_Ordinary_Differential_Equations.pdf}{\textbf{Homogeneous boundary conditions}} refer to a set of equations/constraints of the type
\begin{equation}\label{eq:homogeneous boundary conditions}
    \sum_{k=1}^n (M_{jk}x^{(k-1)}(a) + N_{jk}x^{(k-1)}(b))=0 \quad (j=1,\ldots,m) 
\end{equation}
where $M_{jk}, N_{jk}$ are complex constants.
\end{defn}

\begin{defn}
    A \href[page=296]{./textbooks/[Earl_A_Coddington_Norman_Levinson]_Theory_of_Ordinary_Differential_Equations.pdf}{\textbf{homogeneous boundary-value problem}} concerns finding the solutions of 
    \[Lx=0\] on $[a,b]$ which satisfy some homogeneous boundary conditions defined above.
\end{defn}

\begin{defn}
    For any homogeneous boundary value problem, an \href[page=296]{./textbooks/[Earl_A_Coddington_Norman_Levinson]_Theory_of_Ordinary_Differential_Equations.pdf}{\textbf{adjoint problem}} refers to the problem of finding the solutions of
    \[L^{+}x:=(-1)^n(\bar{p}_0x)^{(n)}+(-1)^{n-1}(\bar{p}_1x)^{(n-1)}+\cdots+\bar{p}_nx=0\]
    on $[a,b]$ which satisfy some homogeneous boundary conditions ``complementary'' to the conditions associated with the solutions of $Lx=0$.
\end{defn}

\begin{thm}{\href[page=296]{./textbooks/[Earl_A_Coddington_Norman_Levinson]_Theory_of_Ordinary_Differential_Equations.pdf}{(Green's formula)}}
    For $u, v\in C^n$ on $[a,b]$,
    \begin{equation}\label{eq:green's formula}
        \int_{t_1}^{t_2}(Lu)\bar{v}\,dt - \int_{t_1}^{t_2}u(\bar{L^+v})\,dt = [uv](t_2) - [uv](t_1) 
    \end{equation}
    where $a\leq t_1<t_2\leq b$ and $[uv](t)$ is the form in $(u, u', \ldots, u^{(n-1)})$ and $(v, v', \ldots, v^{(n-1)})$ given by
    \[[uv](t)=\sum_{m=1}^n\sum_{j+k=m-1}(-1)^i u^{(k)}(t)(p_{n-m}\bar{v})^{(j)}(t).\]
\end{thm}
\begin{rmk}
    \href[page=297]{./textbooks/[Earl_A_Coddington_Norman_Levinson]_Theory_of_Ordinary_Differential_Equations.pdf}{Alternatively}, $[uv](t)$ can be written as {\color{blue}(checked for $n=2$)}
    \begin{equation}\label{eq:[uv](t) in B matrix}
        [uv](t)=\sum_{j,k=1}^n B_{jk}(t)u^{(k-1)}(t)\bar{v}^{(j-1)}(t) 
    \end{equation}
        where $B_{jk}$ are the $j, k$-entry of the $n\times n$ matrix
    \[
        B(t)=\begin{bmatrix}
            B_{11} & B_{12} & \cdots & \cdots & p_0(t)\\
            \vdots & \vdots & \cdots & -p_0(t) & 0\\
            (-1)^{n-1}p_0(t) & 0 & \cdots & 0 & 0.
        \end{bmatrix}
    \]
    Since $B(t)$ is square with $\det B(t)=(p_0(t))^n$ where $p_0(t)\neq 0$ on $[a,b]$ (as in the definition of $L$), $B(t)$ is nonsingular/invertible for $t\in [a,b]$.
\end{rmk}

\begin{defn}
    For vectors $f=(f_1,\ldots,f_k)$, $g=(g_1,\ldots,g_k)$, define the product
    \[f\cdot g:=\sum_{i=1}^k f_i\bar{g}_i.\]
\end{defn}

\begin{defn}
    A \href[page=297]{./textbooks/[Earl_A_Coddington_Norman_Levinson]_Theory_of_Ordinary_Differential_Equations.pdf}{\textbf{semibilinear form}} is a complex-valued function $\mathcal{S}$ defined for pairs of vectors $f=(f_1,\ldots,f_k)$, $g=(g_1,\ldots,g_k)$ satisfying
    \begin{align*}
        \mathcal{S}(\alpha f+\beta g, h)&=\alpha\mathcal{S}(f,h) + \beta\mathcal{S}(g,h)\\
        \mathcal{S}(f, \alpha g + \beta h) &= \bar{\alpha}\mathcal{S}(f,g) + \bar{\beta}\mathcal{S}(f, h)
    \end{align*}
    for any complex numbers $\alpha, \beta$ and vectors $f,g,h$.

    {\color{blue}Note that $\mathcal{S}$ is linear in the first argument but not the second one. If $\mathcal{S}$ were bilinear, it would be linear in each argument.}
\end{defn}
\begin{rmk}
    If
    \[S = \begin{bmatrix}
        s_{11} & \cdots & s_{1k}\\
        \vdots &  & \vdots\\
        s_{k1} & \cdots & s_{kk}
    \end{bmatrix},\]
    then $Sf\cdot g$ is a semibilinear form
    {\color{blue}    
    \begin{equation}\label{eq:semibilinear form}
        \begin{split}
        \mathcal{S}(f,g) &:= Sf\cdot g\\
        &= \begin{bmatrix}
            s_{11} & \cdots & s_{1k}\\
            \vdots & & \vdots\\
            s_{k1} & \cdots & s_{kk}
        \end{bmatrix} \begin{bmatrix}
            f_1\\
            \vdots\\
            f_k
        \end{bmatrix} \cdot \begin{bmatrix}
            g_1\\
            \vdots\\
            g_k
        \end{bmatrix}\\
        &= \begin{bmatrix}
            \sum_{j=1}^k s_{1j}f_j\\
            \vdots\\
            \sum_{j=1}^k s_{kj}f_j
        \end{bmatrix}\cdot \begin{bmatrix}
            g_1\\
            \vdots\\
            g_k
        \end{bmatrix}\\
        &= \sum_{i=1}^k\left(\sum_{j=1}^k s_{ij}f_j\right)\bar{g}_i\\
        &=\sum_{i,j=1}^k s_{ij}f_i\bar{g}_i. 
        \end{split}
    \end{equation}
    Indeed:
    \begin{align*}
        \mathcal{S}(\alpha f+\beta g, h)
        &= \sum_{i,j=1}^k s_{ij}(\alpha f_j + \beta g_j)\bar{h}_i\\
        &= \alpha \sum_{i,j=1}^k s_{ij}f_j\bar{h}_i + \beta \sum_{i,j=1}^k g_j\bar{h}_i\\
        &= \alpha Sf\cdot h + \beta Sg\cdot h\\
        &= \alpha\mathcal{S}(f,h) + \beta\mathcal{S}(g,h).
    \end{align*}
    Similarly,
    \begin{align*}
        \mathcal{S}(f, \alpha g + \beta h)
        &= \sum_{i,j=1}^k s_{ij}f_j(\bar{\alpha g_i + \beta h_i})\\
        &= \bar{\alpha}\sum_{i,j=1}^k s_{ij}f_j\bar{g}_i + \bar{\beta}\sum_{i,j=1}^k f_j \bar{h}_i\\
        &= \bar{\alpha}Sf\cdot g + \bar{\beta}Sf\cdot h\\
        &= \bar{\alpha}\mathcal{S}(f,g) + \bar{\beta}\mathcal{S}(f,h).
    \end{align*}
    }
\end{rmk}

\begin{rmk}
    Under a similar matrix framework, we see that $[uv](t)$ is a semibilinear form with matrix $B(t)$:

    {\color{blue}
    Let $\vec{u}=(u, u', \ldots, u^{(n-1)})$ and $\vec{v}=(v, v', \ldots, v^{(n-1)})$. Then we have
    \begin{align*}
        [uv](t) &= \sum_{j,k=1}^n B_{jk}(t)u^{(k-1)}(t)\bar{v}^{(j-1)}(t)\quad\mbox{(by \eqref{eq:[uv](t) in B matrix})}\\
        &= \sum_{i,j=1}^n (B_{ij}u^{(j-1)}\overline{v}^{(i-1)})(t)\\
        &= (B\vec{u}\cdot \vec{v})(t)\\
        &= \mathcal{S}(\vec{u},\vec{v})(t).
    \end{align*}
    With this notation, we can rewrite the right hand side of Green's formula as
    \begin{align*}
        [uv](t_2)-[uv](t_1) &= \sum_{j,k=1}^n B_{jk}(t_2)u^{(k-1)}(t_2)\bar{v}^{(j-1)}(t_2) - \sum_{j,k=1}^n B_{jk}(t_2)u^{(k-1)}(t_1)\bar{v}^{(j-1)}(t_1)\\
        &= B(t_2)\vec{u}(t_2)\cdot \vec{v}(t_2) - B(t_1)\vec{u}(t_1)\cdot \vec{v}(t_1)\\
        &= \begin{bmatrix}
            B_{11}(t_2) & \cdots & B_{1n}(t_2)\\
            \vdots &  & \vdots\\
            B_{n1}(t_2) & \ldots & B_{nn}(t_2)
        \end{bmatrix} 
        \begin{bmatrix}
        u(t_2)\\
        \vdots\\
        u^{(n-1)}(t_2)
        \end{bmatrix}\cdot 
        \begin{bmatrix}
            \bar{v}(t_2)\\
            \vdots\\
            \bar{v}^{(n-1)}(t_2)
        \end{bmatrix} -\\
        &\qquad \begin{bmatrix}
            B_{11}(t_1) & \cdots & B_{1n}(t_1)\\
            \vdots &  & \vdots\\
            B_{n1}(t_1) & \ldots & B_{nn}(t_1)
        \end{bmatrix} 
        \begin{bmatrix}
        u(t_1)\\
        \vdots\\
        u^{(n-1)}(t_1)
        \end{bmatrix}\cdot 
        \begin{bmatrix}
            \bar{v}(t_1)\\
            \vdots\\
            \bar{v}^{(n-1)}(t_1)
        \end{bmatrix}\\
        &= \begin{bmatrix}
            -B_{11}(t_1) & \cdots & -B_{1n}(t_1)\\
            \vdots &  & \vdots\\
            -B_{n1}(t_1) & \ldots & -B_{nn}(t_1)
        \end{bmatrix} 
        \begin{bmatrix}
        u(t_1)\\
        \vdots\\
        u^{(n-1)}(t_1)
        \end{bmatrix}\cdot 
        \begin{bmatrix}
            \bar{v}(t_1)\\
            \vdots\\
            \bar{v}^{(n-1)}(t_1)
        \end{bmatrix} + \\
        &\qquad \begin{bmatrix}
            B_{11}(t_2) & \cdots & B_{1n}(t_2)\\
            \vdots &  & \vdots\\
            B_{n1}(t_2) & \ldots & B_{nn}(t_2)
        \end{bmatrix} 
        \begin{bmatrix}
        u(t_2)\\
        \vdots\\
        u^{(n-1)}(t_2)
        \end{bmatrix}\cdot 
        \begin{bmatrix}
            \bar{v}(t_2)\\
            \vdots\\
            \bar{v}^{(n-1)}(t_2)
        \end{bmatrix}\\
        &= \begin{bmatrix}
            -B_{11}(t_1) & \cdots & -B_{1n}(t_1) & 0 & \cdots & 0\\
            \vdots &  & \vdots & \vdots &  & \vdots\\
            -B_{n1}(t_1) & \cdots & -B_{nn}(t_1) & 0 & \cdots & 0\\
            0 & \cdots & 0 & B_{11}(t_2) & \cdots & B_{1n}(t_2)\\
            \vdots &  & \vdots & \vdots &  & \vdots\\
            0 & \cdots & 0 & B_{n1}(t_2) & \cdots & B_{nn}(t_2)
        \end{bmatrix} 
        \begin{bmatrix}
            u(t_1)\\
            \vdots\\
            u^{(n-1)}(t_1)\\
            u(t_2)\\
            \vdots\\
            u^{(n-1)}(t_2)
        \end{bmatrix}\cdot
        \begin{bmatrix}
            \bar{v}(t_1)\\
            \vdots\\
            \bar{v}^{(n-1)}(t_1)\\
            \bar{v}(t_2)\\
            \vdots\\
            \bar{v}^{(n-1)}(t_2)
        \end{bmatrix}\\
        &= \begin{bmatrix}
            -B(t_1) & 0_n\\
            0_n & B(t_2)
        \end{bmatrix}
        \begin{bmatrix}
            u(t_1)\\
            \vdots\\
            u^{(n-1)}(t_1)\\
            u(t_2)\\
            \vdots\\
            u^{(n-1)}(t_2)
        \end{bmatrix}\cdot
        \begin{bmatrix}
            \bar{v}(t_1)\\
            \vdots\\
            \bar{v}^{(n-1)}(t_1)\\
            \bar{v}(t_2)\\
            \vdots\\
            \bar{v}^{(n-1)}(t_2)
        \end{bmatrix}\\
        &=:\hat{B}
        \begin{bmatrix}
            \vec{u}(t_1)\\
            \vec{u}(t_2)
        \end{bmatrix}\cdot
        \begin{bmatrix}
            \vec{v}(t_1)\\
            \vec{v}(t_2)
        \end{bmatrix}.
    \end{align*}
    }
    Since \href[page=298]{./textbooks/[Earl_A_Coddington_Norman_Levinson]_Theory_of_Ordinary_Differential_Equations.pdf}{$\det\hat{B}=(-1)^n\det B(t_1)\det B(t_2)$}\unsure{Why the $(-1)^n$?}, $\hat{B}$ is nonsingular for $t_1, t_2\in [a,b]$ {\color{blue}(since $B(t)$ is nonsingular for $t\in [a,b]$, as shown before)}.
\end{rmk}

\subsubsection{Boundary form formula}
\begin{defn}\label{defn:boundary form}
    Given any set of $2mn$ complex constants $M_{ij}, N_{ij}$ ($i=1,\ldots, m;\,j=1,\ldots,n$), define $m$ \href[page=298]{./textbooks/[Earl_A_Coddington_Norman_Levinson]_Theory_of_Ordinary_Differential_Equations.pdf}{\textbf{boundary operators (boundary forms)}} $U_1,\ldots,U_m$ for functions $x$ on $[a,b]$, for which $x^{(j)}$ ($j=1,\ldots,n-1$) exists at $a$ and $b$, by
    \begin{equation}\label{eq:U_i defn}
        U_i x = \sum_{j=1}^n (M_{ij}x^{(j-1)}(a) + N_{ij}x^{(j-1)}(b))\quad\mbox{($i=1,\ldots,m$)} 
    \end{equation}
    $U_i$ are \textbf{linearly independent} if the only set of complex constants $c_1, \ldots, c_m$ for which
    \[\sum_{i=1}^m c_i U_ix=0\]
    for all $x\in C^{n-1}$ on $[a,b]$ is $c_1=c_2=\cdots =c_m=0$.
\end{defn}
\begin{rmk}
    Note that for $\alpha, \beta\in\C$ and $x_1, x_2\in C^{n-1}$ on $[a,b]$,
    {\color{blue}
    \begin{align*}
        U_i(\alpha x_1+\beta x_2) &= \sum_{j=1}^n (M_{ij}(\alpha x_1 + \beta x_2)^{(j-1)}(a) + N_{ij}(\alpha x_1 + \beta x_2)^{(j-1)}(b))\\
        &= \alpha \sum_{j=1}^n (M_{ij}x_1^{(j-1)}(a) + N_{ij}x_1^{(j-1)}(b)) + \beta \sum_{j=1}^n (M_{ij}x_2^{(j-1)}(a) + N_{ij}x_2^{(j-1)}(b))\quad\mbox{(by linearity of derivatives)}\\
        &= \alpha U_i x_1 + \beta U_i x_2.
    \end{align*}
    }
    So $U_i$ are linear operators.
\end{rmk}

\begin{rmk}
    To describe \eqref{eq:U_i defn} with matrices, define
    \[
        \xi:= \begin{bmatrix}x\\ x'\\ \vdots \\ x^{(n-1)}
        \end{bmatrix};
            \quad
        U := \begin{bmatrix}U_1\\ U_2\\ \vdots \\ U_m
        \end{bmatrix};
            \quad
        M := \begin{bmatrix}
            M_{11} & \cdots & M_{1n}\\
            \vdots &  & \vdots\\
            M_{m1} & \cdots & M_{mn}
        \end{bmatrix};
        \quad
        N := \begin{bmatrix}
            N_{11} & \cdots & N_{1n}\\
            \vdots &  & \vdots\\
            N_{m1} & \cdots & N_{mn}
        \end{bmatrix}.
    \]
    Then \eqref{eq:U_i defn} can be written as
    \[Ux = M\xi(a) + N\xi(b).\]
    {\color{blue}
    Indeed:
    \begin{align*}
        M\xi(a) + N\xi(b) &= \begin{bmatrix}
            M_{11} & \cdots & M_{1n}\\
            \vdots &  & \vdots\\
            M_{m1} & \cdots & M_{mn}
        \end{bmatrix}\begin{bmatrix}x(a)\\ x'(a)\\ \vdots \\ x^{(n-1)}(a)
            \end{bmatrix} + \begin{bmatrix}
                N_{11} & \cdots & N_{1n}\\
                \vdots &  & \vdots\\
                N_{m1} & \cdots & N_{mn}
            \end{bmatrix}\begin{bmatrix}x(b)\\ x'(b)\\ \vdots \\ x^{(n-1)}(b)
            \end{bmatrix}\\
            &= \begin{bmatrix}
                \sum_{j=1}^n M_{1j}x^{(j-1)}(a)\\
                \vdots\\
                \sum_{j=1}^n M_{mj}x^{(j-1)}(a)
            \end{bmatrix} + \begin{bmatrix}
                \sum_{j=1}^n N_{1j}x^{(j-1)}(b)\\
                \vdots\\
                \sum_{j=1}^n N_{mj}x^{(j-1)}(b)
            \end{bmatrix}\\
            &= \begin{bmatrix}
                \sum_{j=1}^n (M_{1j}x^{(j-1)}(a) + N_{1j}x^{(j-1)}(b))\\
                \vdots\\
                \sum_{j=1}^n (M_{mj}x^{(j-1)}(a) + N_{mj}x^{(j-1)}(b))
            \end{bmatrix}\\
            &= \begin{bmatrix}
                U_1 x\\
                \vdots\\
                U_m x
            \end{bmatrix} = \begin{bmatrix}U_1\\ U_2\\ \vdots \\ U_m
            \end{bmatrix}x = Ux.
    \end{align*}
    }
    Define the $m\times 2n$ matrix
    \[(M:N):=\begin{bmatrix}
        M_{11} & \cdots & M_{1n} & N_{11} & \cdots & N_{1n}\\
        \vdots &  & \vdots & \vdots & & \vdots\\
        M_{m1} & \cdots & M_{mn} & N_{m1} & \cdots & N_{mn}
    \end{bmatrix}.\]
    Then $U_1,\ldots,U_m$ are linearly independent if and only if $\rank(M:N)=m$, or equivalently, $\rank(U)=m$. 
    {\color{blue}
    Recall that the rank of a matrix is the largest number of linearly independent rows or columns in it. For a matrix $A_{m\times n}$, $\rank(A)\leq \min\{m,n\}$ and $\rank(A)=\rank(A^T)$.

    $Ux$ can be written as
    \begin{align*}
        Ux &= \begin{bmatrix}
            \sum_{j=1}^n (M_{1j}x^{(j-1)}(a) + N_{1j}x^{(j-1)}(b))\\
            \vdots\\
            \sum_{j=1}^n (M_{mj}x^{(j-1)}(a) + N_{mj}x^{(j-1)}(b))
        \end{bmatrix}\\
        &= \begin{bmatrix}
            M_{11} & \cdots & M_{1n} & N_{11} & \cdots & N_{1n}\\
            \vdots &  & \vdots & \vdots & & \vdots\\
            M_{m1} & \cdots & M_{mn} & N_{m1} & \cdots & N_{mn}
        \end{bmatrix} \begin{bmatrix}x(a)\\\vdots\\x^{(n-a)}(a)\\ x(b)\\\vdots\\x^{(n-1)}(b)\end{bmatrix}\\
        &= (M:N)\begin{bmatrix}
            \xi(a)\\
            \xi(b)
        \end{bmatrix}.
    \end{align*}
    }
\end{rmk}

\begin{defn}\label{defn:complementary boundary form}
    If $U=(U_1,\ldots, U_m)$ is any boundary form with $\rank(U)=m$ and $U_c=(U_{m+1},\ldots,U_{2n})$ any form with $\rank(U_c)=2n-m$ such that $(U_1,\ldots, U_{2n})$ has rank $2n$, then $U$ and $U_c$ are \href[page=299]{./textbooks/[Earl_A_Coddington_Norman_Levinson]_Theory_of_Ordinary_Differential_Equations.pdf}{\textbf{complementary boundary forms}}. ``Adjoining'' $U_{m+1},\ldots, U_{2n}$ to $U_1,\ldots,U_m$ is equivalent to imbedding the matrix $(M:N)$ in a $2n\times 2n$ nonsingular matrix {\color{blue}(recall that for square matrices, nonsingular $\iff$ full rank)}.
\end{defn}

We wish to describe the right hand side of Green's formula \eqref{eq:green's formula} as a linear combination of a boundary form $U$ and a complementary form $U_c$. To do so, we consider the following results about the semibilinear form \eqref{eq:semibilinear form}.

\begin{defn}
    For a matrix $A=(a_{ij})$, its \href[page=299]{./textbooks/[Earl_A_Coddington_Norman_Levinson]_Theory_of_Ordinary_Differential_Equations.pdf}{\textbf{adjoint}} is defined as the conjugate transpose $A^* = (\bar{a}_{ij})$.
\end{defn}
\begin{prop}
    In the context of the semibilinear form \eqref{eq:semibilinear form}, we have
    \begin{equation}\label{eq:semibilinear adjoint}
        Sf\cdot g = f\cdot S^*g. 
    \end{equation}
\end{prop}
{\color{blue}
    \begin{proof}
        \begin{align*}
            Sf\cdot g &= \sum_{i,j=1}^k s_{ij}f_j\bar{g}_i \quad\mbox{(by \eqref{eq:semibilinear form})};\\
            f\cdot S^*g &= \begin{bmatrix}
                f_1\\
                \vdots\\
                f_k
            \end{bmatrix}\cdot \begin{bmatrix}
                \bar{s}_{11} & \cdots & \bar{s}_{k1}\\
                \vdots & & \vdots\\
                \bar{s}_{1k} & \cdots & \bar{s}_{kk}
            \end{bmatrix} \begin{bmatrix}
                g_1\\
                \vdots\\
                g_k
            \end{bmatrix}\\
            &= \begin{bmatrix}
                f_1\\
                \vdots\\
                f_k
            \end{bmatrix}\cdot \begin{bmatrix}
                \sum_{j=1}^k \bar{s}_{j1}g_j\\
                \vdots\\
                \sum_{j=1}^k \bar{s}_{jk}g_j
            \end{bmatrix}\\
            &= \sum_{i=1}^k f_i\cdot \overline{\left(\sum_{j=1}^k\bar{s}_{ji}g_j\right)}\\
            &= \sum_{i=1}^k f_i\cdot \left(\sum_{j=1}^k s_{ji}\bar{g}_j\right)\\
            &= \sum_{i,j=1}^k s_{ji}f_i\bar{g}_j = Sf\cdot g.
        \end{align*}
    \end{proof}
}

\begin{prop}\label{prop:unique nonsingular G in semibilinear form}
    Let $\mathcal{S}$ be the semibilinear form associated with a nonsingular matrix $S$. Suppose $\bar{f}:=Ff$ where $F$ is a nonsingular matrix. Then there exists a unique nonsingular matrix $G$ such that if $\bar{g}=Gg$, then $\mathcal{S}(f,g)=\bar{f}\cdot \bar{g}$ for all $f, g$.
\end{prop}
{\color{blue}
\begin{proof}
    Let $G:=(SF^{-1})^*$, then
    \begin{align*}
        \mathcal{S}(f,g) &= Sf\cdot g\\
        &= S(F^{-1}F)f\cdot g\\
        &= SF^{-1}(Ff)\cdot g\\
        &= SF^{-1}\bar{f} \cdot g\\
        &= \bar{f}\cdot (SF^{-1})^*g \quad\mbox{(by \eqref{eq:semibilinear adjoint})}\\
        &= \bar{f}\cdot G*g\\
        &= \bar{f}\cdot \bar{g}.
    \end{align*}
    To see that $G$ is nonsingular, note that $\det G = \det((\overline{SF^{-1}})^T) = \det(\overline{SF^{-1}}) = \overline{\det(SF^{-1})} = \overline{\det(S)\det(F)^{-1}}\neq 0$ since $S, F$ are nonsingular.
\end{proof}
}

\begin{prop}
    Suppose $\mathcal{S}$ is associated with the unit matrix $E$, i.e., $\mathcal{S}(f,g)=f\cdot g$. Let $F$ be a nonsingular matrix such that the first $j$ ($1\leq j<k$) components of $\bar{f}=Ff$ are the same as those of $f$. Then the unique nonsingular matrix $G$ such that $\bar{g}=Gg$ and $\bar{f}\cdot \bar{g}=f\cdot g$ (as in Proposition \ref{prop:unique nonsingular G in semibilinear form}) is such that the last $k-j$ components of $\bar{g}$ are linear combinations of the last $k-j$ components of $g$ with nonsingular coefficient matrix.
\end{prop}
{\color{blue}
\begin{proof}
    We note that for the condition on $F$ to hold, $F$ must have the form
\[\begin{bmatrix}E_j & 0_+\\
F_+ & F_{k-j}\end{bmatrix}_{k\times k}\]
where $E_j$ is the $j\times j$ identity matrix, $0_+$ is the $j\times (k-j)$ zero matrix, $F_+$ is a $(k-j)\times j$ matrix, and $F_{k-j}$ a $(k-j)\times (k-j)$ matrix. Let $G$ be the unique nonsingular matrix in Proposition \ref{prop:unique nonsingular G in semibilinear form}. Write $G$ as
\[\begin{bmatrix}G_j & G_-\\
G_= & G_{k-j}\end{bmatrix}_{k\times k}\]
where $G_j, G_-, G_=, G_{k-j}$ are $j\times j, j\times (k-j), (k-j)\times j, (k-j)\times (k-j)$ matrices, respectively. By the definition of $G$,
\[f\cdot g = Ff\cdot Gg = \bar{f}\cdot Gg = G^*\bar{f}\cdot g = G^*Ff\cdot g,\]
(where the third equality follows from a reverse application of \eqref{eq:semibilinear adjoint} with $\bar{f}$ as $f$, $G^*$ as $S$) which implies
\[G^*F = E_k.\]
Since
\begin{align*}
    G^*F &= \begin{bmatrix}
        G^*_j & G^*_=\\
        G^*_- & G^*_{k-j}
    \end{bmatrix}\begin{bmatrix}E_j & 0_+\\
    F_+ & F_{k-j}\end{bmatrix}\\
    &= \begin{bmatrix}
        G^*_j + G^*_= F_+ & G^*_= F_{k-j}\\
        G^*_- + G^*_{k-j} F_+ & G^*_{k-j}F_{k-j}
    \end{bmatrix}\\
    &= \begin{bmatrix}
        E_j & 0_{j\times (k-j)}\\
        0_{(k-j)\times j} & E_{k-j}
    \end{bmatrix}.
\end{align*}
Thus, $G^*_=F_{k-j}=0_+$, the $j\times (k-j)$ zero matrix. But $\det F = \det(E_j)\cdot \det(F_{k-j})\neq 0$, so $\det F_{k-j}\neq 0$ and we must have $G^*_==0_+$, i.e., $G_= =0_{(k-j)\times j}$. Thus, $G$ is upper-triangular, and so $\det G = \det G_j \cdot \det G_{k-j}\neq 0$, which implies $\det G_{k-j}\neq 0$ and $G_{k-j}$ is nonsingular. Hence,
\[\bar{g} = Gg = \begin{bmatrix}G_j & G_-\\
0_{(k-j)\times j} & G_{k-j}\end{bmatrix} \begin{bmatrix}
g_1\\
\vdots\\
g_k
\end{bmatrix}\]
where $G_{k-j}$ is the nonsingular coefficient matrix such that
\[\begin{bmatrix}
    \bar{g}_{j-1}\\
    \vdots\\
    \bar{g}_{k}
\end{bmatrix} = G_{k-j}\begin{bmatrix}
    g_{j-1}\\
    \vdots\\
    g_{k}
\end{bmatrix}.\]
\end{proof}
}

\begin{thm}\href[page=300]{./textbooks/[Earl_A_Coddington_Norman_Levinson]_Theory_of_Ordinary_Differential_Equations.pdf}{(Boundary-form formula)}\label{thm:boundary form formula}
    Given any boundary form $U$ of rank $m$ (Definition \ref{defn:boundary form}), and any complementary form $U_c$ (Definition \ref{defn:complementary boundary form}), there exist unique boundary forms $U_c^+$, $U^+$ of rank $m$ and $2n-m$, respectively, such that
    \begin{equation}\label{eq:boundary form formula}
        [xy](b)-[xy](a) = Ux\cdot U_c^+y + U_cx\cdot U^+y.
    \end{equation}
    If $\tilde{U}_c$ is any other complementary form to $U$, and $\tilde{U}^+_c, \tilde{U}^+$ the corresponding forms of rank $m$ and $2n-m$, then
    \[\tilde{U}^+ y = C^*U^+y\]
    for some nonsingular matrix $C$.
\end{thm}
\begin{proof}

\end{proof}

\end{document}